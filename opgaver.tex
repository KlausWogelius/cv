\begin{rubric}{Seneste 5 års opgaver}
%
\subrubric{Udvidelse af IP radiokæden fra Uummannaq til Upernavik}
\entry*[okt 2015 - feb 2016] \textbf{Beskrivelse:} Den eksisterende 1 Gbps IP radiokæde skal forlænges fra Uummannaq og videre til Upernavik. Projektet indeholder etablering af 4 nye sites og installation af udstyr på sammenlagt 7 sites. Radiokædeforlængelsen skal afleveres i 2016.
\entry* \textbf{Involverede:} Tele Greenland.
\entry* \textbf{Egen rolle:} Projektdeltager.
%
\subrubric{Forundersøgelse før Udvidelse af kapacitet på IP backbone}
\entry*[sep 2015 - feb 2016] \textbf{Beskrivelse:} Design og planlægning af en opgradering af Tele Greenland's IP backbone router-net tilpasset et nyt søkabel langs Grønland vestkyst.
\entry* \textbf{Involverede:} Tele Greenland, Conscia.
\entry* \textbf{Egen rolle:} Projektleder.
\entry* \textbf{Mio. DKK:} $0,75$
\entry* \textbf{Mandemåneder:} $4,5$
%
\subrubric{Gennemførelse af udbudsforretning for 10 Gbps radiokæde}
\entry*[apr 2015 - aug 2015] \textbf{Beskrivelse:} En 10 Gbps radiokæde planlægges installeret mellem Aasiaat og Ilulissat sommeren 2016. Udbudsmateriale hertil er derfor udarbejdet.
\entry* \textbf{Involverede:} Tele Greenland, NEC, Huawei, Ceragon, Ericsson.
\entry* \textbf{Egen rolle:} udarbejdet udbudsmateriale udfra specifikationer givet af projektlederen. Deltagelse i evaluering af tilbud og valg af leverandør.
%
%\subrubric{Optimering af router-netværk for overvågningssystemer}
%\entry*[maj 2015 - ] \textbf{Beskrivelse:} Et stabilt, fejlredundant og hurtigt OSPF router netværk for overvågningssystemer til diverse transmissionssystemer skal etableres.
%\entry* \textbf{Involverede:} Tele Greenland.
%\entry* \textbf{Egen rolle:} Opstart af opgaven.
%
\subrubric{Anlæg af radiokæde fra Aasiaat til Uummannak (ialt 8 hop)}
\entry*[jan 2014 - jan 2015] \textbf{Beskrivelse:} Transmissionskapaciteten for IP traffik på Grønlands hovedradiokæde tredobledes på strækket fra Aasiaat til Uummannak.
\entry* \textbf{Involverede:} Tele Greenland, Cisco, Conscia, Ericsson, Air Greenland
\entry* \textbf{Egen rolle:} Projektleder for installation af radiokæde incl. indkøb, logistik, test, installation, idriftssættelse,trafikomlægning og aflevering.
\entry* \textbf{Mio. DKK:} $10$
\entry* \textbf{Mandemåneder:} $14$
%
\subrubric{Anlæg af radiokæde fra Nuuk til Aasiaat (ialt 17 hop)}
\entry*[jan - dec 2013] \textbf{Beskrivelse:} Transmissionskapaciteten for IP traffik på Grønlands hovedradiokæde tredobledes på strækket fra Nuuk til Aasiaat.
\entry* \textbf{Involverede:} Tele Greenland, Cisco, Conscia, Ericsson, Air Greenland
\entry* \textbf{Egen rolle:} Projektleder for installation af radiokæde incl. kontrakt, indkøb, logistik, test, installation, idriftssættelse, uddannelse af driftpersonale og aflevering.
\entry* \textbf{Mio. DKK:} $15$
\entry* \textbf{Mandemåneder:} $20$
%
\subrubric{Omlægning af radiokædetrafikken til All-IP}
\entry*[jan - mar 2012] \textbf{Beskrivelse:} En udvidelse af Grønlands hovedradiokæde fra Nuuk og nordover planlægges
\entry* \textbf{Involverede:} Tele Greenland, Cisco, Ericsson
\entry* \textbf{Egen rolle:} Projektleder for teknisk planlægning af radiokæde incl. redesign og integration af Teles routernet.
\entry* \textbf{Mio. DKK:} $0,4$
\entry* \textbf{Mandemåneder:} $2$
%
\subrubric{Udvidelse af bygders IP fra 2Mb til 45Mb}
\entry*[jan - mar 2012] \textbf{Beskrivelse:} Omkonfigurering af eksisterende udstyr.
\entry* \textbf{Involverede:} Tele Greenland, Cisco, Ericsson
\entry* \textbf{Egen rolle:} Tovholder på projektet.
\entry* \textbf{Mio. DKK:} $0,3$
\entry* \textbf{Mandemåneder:} $2$
%
\subrubric{Omlægning af hovedradiokæden nær Sisimiut}
\entry*[mar - nov 2011] \textbf{Beskrivelse:} Ødestationen DYET er et svært tilgængeligt, sårbart og driftsomkostningstungt knudepunkt på Tele Greenlands hovedradiokæde. Derfor ønskede man at undersøge om det kunne betale sig at etablere en alternativ fremføringsvej.
\entry* \textbf{Involverede:} Tele Greenland, Nukissiorfiit, Air Greenland
\entry* \textbf{Egen rolle:} Udarbejdelse af teknisk og økonomisk udredning.
\entry* \textbf{Mio. DKK:} $0,4$
\entry* \textbf{Mandemåneder:} $3,5$
%
\subrubric{Beskyttelse af Tele Greenlands internationale søkabel mod brud fra is}
\entry*[jan - jul 2011] \textbf{Beskrivelse:} Efter at isfjelde gentagne gange havde overskåret søkablet ud for Nuuk besluttedes det at lave en effektiv sikring mod fremtidige brud.
\entry* \textbf{Involverede:} Tele Greenland, Aarsleff, V\&S Hanab, Kompetancekompagniet, Råstoffor-valtningen, Sermersooq kommunia, Nationalmuseet, Nuna minerals, Alcatel-Lucent, Grønlands arbejdsgiverforening, Sound and Sea Technology, Nuna advokater, Bureau of Minerals and Petroleum m.f
\entry* \textbf{Egen rolle:} Udarbejdelse af teknisk løsning, kravsspec. og udbudsmateriale samt indhentning af tilbud. 
\entry* \textbf{Mio. DKK:} $3$ (selve anlægsprojektet kostede 65 mio DKK).
\entry* \textbf{Mandemåneder:} $6$
%
\subrubric{Udskiftning af 4 bygde-links}
\entry*[apr 2011 - okt 2012] \textbf{Beskrivelse:} 4 bygdeforbindelser skulle udvides vha. en ny radiokæde af typen Ericsson minilink.
\entry* \textbf{Involverede:} Tele Greenland, Ericsson, Air Greenland
\entry* \textbf{Egen rolle:} Tovholder på lille anlægsopgave.
\entry* \textbf{Mio. DKK:} $1$
\entry* \textbf{Mandemåneder:} $3$
%
%\subrubric{Udskiftning af MF-sendere i Grønland}
%\entry*[jan - maj 2010] \textbf{Beskrivelse:} Da reservedele til  de gamle MF-sendere i Grønland ikke længere kunne skaffes onskede ledelsen at udskifte til nyt udstyr. 
%\entry* \textbf{Involverede:} Tele Greenland, Japan Radio Co., Rohde \& Schwarz, Naviair
%\entry* \textbf{Egen rolle:} Udarbejdelse af kravsspec og udbudsmateriale samt indhentning af tilbud.
%\entry* \textbf{Mio. DKK:} $0,2$
%\entry* \textbf{Mandemåneder:} $4$
%%
%\subrubric{Måling af latency på søkablet}
%\entry*[okt 2007] \textbf{Beskrivelse:} Teknisk undersøgelse
%\entry* \textbf{Involverede:} Tele Greenland
%\entry* \textbf{Egen rolle:} Tovholder på opgaven og udarbejdelse af rapport.
%\entry* \textbf{Mio. DKK:} $0,05$
%\entry* \textbf{Mandemåneder:} $1$
%%
%\subrubric{Udskiftning af Grønlands hovedradiokæde}
%\entry*[jan 2007 - dec 2008] \textbf{Beskrivelse:} Tele Greenlands hovedradiokæde (udstyr i 26 repeaterhytter og 24 terminalbygninger) blev udskiftet fra Siemens PDH-udstyr til Marconi SDH-udstyr.
%\entry* \textbf{Involverede:} Tele Greenland, Marconi, Siemens, Panduit
%\entry* \textbf{Egen rolle:} Detailprojektering og ledelse af anlægsopgave.
%\entry* \textbf{Mio. DKK:} $20$
%\entry* \textbf{Mandemåneder:} $16$
%%
%\subrubric{Søkabelterminaler}
%\entry*[jun 2008 - maj 2009] \textbf{Beskrivelse:} Installation af terminaler til Tele Greenlands internationale Søkabel. SDH-multiplexere blev installeret på Island, i Qaqortoq, i Nuuk, på New Foundland og i København. Pånær nogle satellitforbindelser går al transmission til og fra Grønland samt udlejede forbindelser til trediepart fra Canada til Europa gennem disse terminaler.
%\entry* \textbf{Involverede:} Tele Greenland, Nokia-Siemens, Farice (Island), TDC, diverse operatører i USA, England og Canada.
%\entry* \textbf{Egen rolle:} Projektmedarbejder. Deltog i systemdesign, fabrikstest, systemtest, idriftsættelse og aflevering.
%\entry* \textbf{Mio. DKK:} $7$
%\entry* \textbf{Mandemåneder:} $15$
%%
%\subrubric{Kvalitets- og kapacitetskrav til grønlandske radiokædeforbindelser}
%\entry*[jun - dec 2006] \textbf{Beskrivelse:} En teknisk udredning. Efter udredningen besluttede Tele at:
%\begin{compactenum} %kompakt "enumerate" fra pakken "paralist"
%\item Anlægge et søkabel fra Island til Qaqortoq - fra Qaqortoq til Nuuk og fra Nuuk til New Foundland.
%\item Omdanne Grønlands hovedradiokæde fra SDH til All-IP.
%\item Udvide Grønlands hovedradiokæde yderligere 330 km frem til Upernavik og nærliggende bygder.
%\end{compactenum}
%\entry* \textbf{Involverede:} Tele Greenland
%\entry* \textbf{Egen rolle:}  Udarbejdelse af teknisk udredning.
%\entry* \textbf{Mio. DKK:} $0,3$
%\entry* \textbf{Mandemåneder:} $6$
%%
%\subrubric{Etablering af Ethernet Backbone i Qaqortoq}
%\entry*[apr - jun 2006] \textbf{Beskrivelse:} En teknisk og økonomisk undersøgelse. Redegørelsen bevirkede at Tele Greenland anlagde en lysleder-ring i Qaqortoq by i sydgrønland.
%\entry* \textbf{Involverede:} Tele Greenland
%\entry* \textbf{Egen rolle:}  Gennemførelse af teknisk og økonomisk undersøgelse og udarbejdelse af redegørelse.
%\entry* \textbf{Mio. DKK:} $0,1$
%\entry* \textbf{Mandemåneder:} $2$
\end{rubric}
